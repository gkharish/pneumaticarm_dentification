
%% bare_conf.tex
%% V1.3
%% 2007/01/11
%% by Michael Shell
%% See:
%% http://www.michaelshell.org/
%% for current contact information.
%%
%% This is a skeleton file demonstrating the use of IEEEtran.cls
%% (requires IEEEtran.cls version 1.7 or later) with an IEEE conference paper.
%%
%% Support sites:
%% http://www.michaelshell.org/tex/ieeetran/
%% http://www.ctan.org/tex-archive/macros/latex/contrib/IEEEtran/
%% and
%% http://www.ieee.org/

%%*************************************************************************
%% Legal Notice:
%% This code is offered as-is without any warranty either expressed or
%% implied; without even the implied warranty of MERCHANTABILITY or
%% FITNESS FOR A PARTICULAR PURPOSE! 
%% User assumes all risk.
%% In no event shall IEEE or any contributor to this code be liable for
%% any damages or losses, including, but not limited to, incidental,
%% consequential, or any other damages, resulting from the use or misuse
%% of any information contained here.
%%
%% All comments are the opinions of their respective authors and are not
%% necessarily endorsed by the IEEE.
%%
%% This work is distributed under the LaTeX Project Public License (LPPL)
%% ( http://www.latex-project.org/ ) version 1.3, and may be freely used,
%% distributed and modified. A copy of the LPPL, version 1.3, is included
%% in the base LaTeX documentation of all distributions of LaTeX released
%% 2003/12/01 or later.
%% Retain all contribution notices and credits.
%% ** Modified files should be clearly indicated as such, including  **
%% ** renaming them and changing author support contact information. **
%%
%% File list of work: IEEEtran.cls, IEEEtran_HOWTO.pdf, bare_adv.tex,
%%                    bare_conf.tex, bare_jrnl.tex, bare_jrnl_compsoc.tex
%%*************************************************************************

% *** Authors should verify (and, if needed, correct) their LaTeX system  ***
% *** with the testflow diagnostic prior to trusting their LaTeX platform ***
% *** with production work. IEEE's font choices can trigger bugs that do  ***
% *** not appear when using other class files.                            ***
% The testflow support page is at:
% http://www.michaelshell.org/tex/testflow/



% Note that the a4paper option is mainly intended so that authors in
% countries using A4 can easily print to A4 and see how their papers will
% look in print - the typesetting of the document will not typically be
% affected with changes in paper size (but the bottom and side margins will).
% Use the testflow package mentioned above to verify correct handling of
% both paper sizes by the user's LaTeX system.
%
% Also note that the "draftcls" or "draftclsnofoot", not "draft", option
% should be used if it is desired that the figures are to be displayed in
% draft mode.
%
\documentclass[conference]{IEEEtran}
\usepackage{blindtext, graphicx}
% Add the compsoc option for Computer Society conferences.
%
% If IEEEtran.cls has not been installed into the LaTeX system files,
% manually specify the path to it like:
% \documentclass[conference]{../sty/IEEEtran}





% Some very useful LaTeX packages include:
% (uncomment the ones you want to load)


% *** MISC UTILITY PACKAGES ***
%
%\usepackage{ifpdf}
% Heiko Oberdiek's ifpdf.sty is very useful if you need conditional
% compilation based on whether the output is pdf or dvi.
% usage:
% \ifpdf
%   % pdf code
% \else
%   % dvi code
% \fi
% The latest version of ifpdf.sty can be obtained from:
% http://www.ctan.org/tex-archive/macros/latex/contrib/oberdiek/
% Also, note that IEEEtran.cls V1.7 and later provides a builtin
% \ifCLASSINFOpdf conditional that works the same way.
% When switching from latex to pdflatex and vice-versa, the compiler may
% have to be run twice to clear warning/error messages.






% *** CITATION PACKAGES ***
%
%\usepackage{cite}
% cite.sty was written by Donald Arseneau
% V1.6 and later of IEEEtran pre-defines the format of the cite.sty package
% \cite{} output to follow that of IEEE. Loading the cite package will
% result in citation numbers being automatically sorted and properly
% "compressed/ranged". e.g., [1], [9], [2], [7], [5], [6] without using
% cite.sty will become [1], [2], [5]--[7], [9] using cite.sty. cite.sty's
% \cite will automatically add leading space, if needed. Use cite.sty's
% noadjust option (cite.sty V3.8 and later) if you want to turn this off.
% cite.sty is already installed on most LaTeX systems. Be sure and use
% version 4.0 (2003-05-27) and later if using hyperref.sty. cite.sty does
% not currently provide for hyperlinked citations.
% The latest version can be obtained at:
% http://www.ctan.org/tex-archive/macros/latex/contrib/cite/
% The documentation is contained in the cite.sty file itself.






% *** GRAPHICS RELATED PACKAGES ***
%
\ifCLASSINFOpdf
  % \usepackage[pdftex]{graphicx}
  % declare the path(s) where your graphic files are
  % \graphicspath{{../pdf/}{../jpeg/}}
  % and their extensions so you won't have to specify these with
  % every instance of \includegraphics
  % \DeclareGraphicsExtensions{.pdf,.jpeg,.png}
\else
  % or other class option (dvipsone, dvipdf, if not using dvips). graphicx
  % will default to the driver specified in the system graphics.cfg if no
  % driver is specified.
  % \usepackage[dvips]{graphicx}
  % declare the path(s) where your graphic files are
  % \graphicspath{{../eps/}}
  % and their extensions so you won't have to specify these with
  % every instance of \includegraphics
  % \DeclareGraphicsExtensions{.eps}
\fi
% graphicx was written by David Carlisle and Sebastian Rahtz. It is
% required if you want graphics, photos, etc. graphicx.sty is already
% installed on most LaTeX systems. The latest version and documentation can
% be obtained at: 
% http://www.ctan.org/tex-archive/macros/latex/required/graphics/
% Another good source of documentation is "Using Imported Graphics in
% LaTeX2e" by Keith Reckdahl which can be found as epslatex.ps or
% epslatex.pdf at: http://www.ctan.org/tex-archive/info/
%
% latex, and pdflatex in dvi mode, support graphics in encapsulated
% postscript (.eps) format. pdflatex in pdf mode supports graphics
% in .pdf, .jpeg, .png and .mps (metapost) formats. Users should ensure
% that all non-photo figures use a vector format (.eps, .pdf, .mps) and
% not a bitmapped formats (.jpeg, .png). IEEE frowns on bitmapped formats
% which can result in "jaggedy"/blurry rendering of lines and letters as
% well as large increases in file sizes.
%
% You can find documentation about the pdfTeX application at:
% http://www.tug.org/applications/pdftex





% *** MATH PACKAGES ***
%
\usepackage[cmex10]{amsmath}
% A popular package from the American Mathematical Society that provides
% many useful and powerful commands for dealing with mathematics. If using
% it, be sure to load this package with the cmex10 option to ensure that
% only type 1 fonts will utilized at all point sizes. Without this option,
% it is possible that some math symbols, particularly those within
% footnotes, will be rendered in bitmap form which will result in a
% document that can not be IEEE Xplore compliant!
%
% Also, note that the amsmath package sets \interdisplaylinepenalty to 10000
% thus preventing page breaks from occurring within multiline equations. Use:
%\interdisplaylinepenalty=2500
% after loading amsmath to restore such page breaks as IEEEtran.cls normally
% does. amsmath.sty is already installed on most LaTeX systems. The latest
% version and documentation can be obtained at:
% http://www.ctan.org/tex-archive/macros/latex/required/amslatex/math/





% *** SPECIALIZED LIST PACKAGES ***
%
\usepackage{algorithmic}
% algorithmic.sty was written by Peter Williams and Rogerio Brito.
% This package provides an algorithmic environment fo describing algorithms.
% You can use the algorithmic environment in-text or within a figure
% environment to provide for a floating algorithm. Do NOT use the algorithm
% floating environment provided by algorithm.sty (by the same authors) or
% algorithm2e.sty (by Christophe Fiorio) as IEEE does not use dedicated
% algorithm float types and packages that provide these will not provide
% correct IEEE style captions. The latest version and documentation of
% algorithmic.sty can be obtained at:
% http://www.ctan.org/tex-archive/macros/latex/contrib/algorithms/
% There is also a support site at:
% http://algorithms.berlios.de/index.html
% Also of interest may be the (relatively newer and more customizable)
% algorithmicx.sty package by Szasz Janos:
% http://www.ctan.org/tex-archive/macros/latex/contrib/algorithmicx/

\usepackage{algorithm}


% *** ALIGNMENT PACKAGES ***
%
\usepackage{array}
% Frank Mittelbach's and David Carlisle's array.sty patches and improves
% the standard LaTeX2e array and tabular environments to provide better
% appearance and additional user controls. As the default LaTeX2e table
% generation code is lacking to the point of almost being broken with
% respect to the quality of the end results, all users are strongly
% advised to use an enhanced (at the very least that provided by array.sty)
% set of table tools. array.sty is already installed on most systems. The
% latest version and documentation can be obtained at:
% http://www.ctan.org/tex-archive/macros/latex/required/tools/


\usepackage{mdwmath}
%\usepackage{mdwtab}
% Also highly recommended is Mark Wooding's extremely powerful MDW tools,
% especially mdwmath.sty and mdwtab.sty which are used to format equations
% and tables, respectively. The MDWtools set is already installed on most
% LaTeX systems. The lastest version and documentation is available at:
% http://www.ctan.org/tex-archive/macros/latex/contrib/mdwtools/


% IEEEtran contains the IEEEeqnarray family of commands that can be used to
% generate multiline equations as well as matrices, tables, etc., of high
% quality.


%\usepackage{eqparbox}
% Also of notable interest is Scott Pakin's eqparbox package for creating
% (automatically sized) equal width boxes - aka "natural width parboxes".
% Available at:
% http://www.ctan.org/tex-archive/macros/latex/contrib/eqparbox/





% *** SUBFIGURE PACKAGES ***
%\usepackage[tight,footnotesize]{subfigure}
% subfigure.sty was written by Steven Douglas Cochran. This package makes it
% easy to put subfigures in your figures. e.g., "Figure 1a and 1b". For IEEE
% work, it is a good idea to load it with the tight package option to reduce
% the amount of white space around the subfigures. subfigure.sty is already
% installed on most LaTeX systems. The latest version and documentation can
% be obtained at:
% http://www.ctan.org/tex-archive/obsolete/macros/latex/contrib/subfigure/
% subfigure.sty has been superceeded by subfig.sty.



%\usepackage[caption=false]{caption}
%\usepackage[font=footnotesize]{subfig}
% subfig.sty, also written by Steven Douglas Cochran, is the modern
% replacement for subfigure.sty. However, subfig.sty requires and
% automatically loads Axel Sommerfeldt's caption.sty which will override
% IEEEtran.cls handling of captions and this will result in nonIEEE style
% figure/table captions. To prevent this problem, be sure and preload
% caption.sty with its "caption=false" package option. This is will preserve
% IEEEtran.cls handing of captions. Version 1.3 (2005/06/28) and later 
% (recommended due to many improvements over 1.2) of subfig.sty supports
% the caption=false option directly:
%\usepackage[caption=false,font=footnotesize]{subfig}
%
% The latest version and documentation can be obtained at:
% http://www.ctan.org/tex-archive/macros/latex/contrib/subfig/
% The latest version and documentation of caption.sty can be obtained at:
% http://www.ctan.org/tex-archive/macros/latex/contrib/caption/




% *** FLOAT PACKAGES ***
%
%\usepackage{fixltx2e}
% fixltx2e, the successor to the earlier fix2col.sty, was written by
% Frank Mittelbach and David Carlisle. This package corrects a few problems
% in the LaTeX2e kernel, the most notable of which is that in current
% LaTeX2e releases, the ordering of single and double column floats is not
% guaranteed to be preserved. Thus, an unpatched LaTeX2e can allow a
% single column figure to be placed prior to an earlier double column
% figure. The latest version and documentation can be found at:
% http://www.ctan.org/tex-archive/macros/latex/base/



%\usepackage{stfloats}
% stfloats.sty was written by Sigitas Tolusis. This package gives LaTeX2e
% the ability to do double column floats at the bottom of the page as well
% as the top. (e.g., "\begin{figure*}[!b]" is not normally possible in
% LaTeX2e). It also provides a command:
%\fnbelowfloat
% to enable the placement of footnotes below bottom floats (the standard
% LaTeX2e kernel puts them above bottom floats). This is an invasive package
% which rewrites many portions of the LaTeX2e float routines. It may not work
% with other packages that modify the LaTeX2e float routines. The latest
% version and documentation can be obtained at:
% http://www.ctan.org/tex-archive/macros/latex/contrib/sttools/
% Documentation is contained in the stfloats.sty comments as well as in the
% presfull.pdf file. Do not use the stfloats baselinefloat ability as IEEE
% does not allow \baselineskip to stretch. Authors submitting work to the
% IEEE should note that IEEE rarely uses double column equations and
% that authors should try to avoid such use. Do not be tempted to use the
% cuted.sty or midfloat.sty packages (also by Sigitas Tolusis) as IEEE does
% not format its papers in such ways.





% *** PDF, URL AND HYPERLINK PACKAGES ***
%
%\usepackage{url}
% url.sty was written by Donald Arseneau. It provides better support for
% handling and breaking URLs. url.sty is already installed on most LaTeX
% systems. The latest version can be obtained at:
% http://www.ctan.org/tex-archive/macros/latex/contrib/misc/
% Read the url.sty source comments for usage information. Basically,
% \url{my_url_here}.





% *** Do not adjust lengths that control margins, column widths, etc. ***
% *** Do not use packages that alter fonts (such as pslatex).         ***
% There should be no need to do such things with IEEEtran.cls V1.6 and later.
% (Unless specifically asked to do so by the journal or conference you plan
% to submit to, of course. )
\usepackage{amsmath}
\usepackage{amsfonts}
\usepackage{amssymb}
\usepackage{savesym}
\savesymbol{AND}
\savesymbol{OR}
\savesymbol{NOT}
\savesymbol{TO}
\savesymbol{COMMENT}
\savesymbol{BODY}
\savesymbol{IF}
\savesymbol{ELSE}
\savesymbol{ELSIF}
\savesymbol{FOR}
\savesymbol{WHILE}

\newcommand\secref[1]{Section\,\ref{#1}}
\newcommand\vx[1]{\mathbf{#1}}
\newcommand\figref[1]{Fig.\,\ref{#1}}
% correct bad hyphenation here
\hyphenation{op-tical net-works semi-conduc-tor}


\begin{document}
%
% paper title
% can use linebreaks \\ within to get better formatting as desired
\title{ILQR implementation on non-linear pneumatic arm actuated by agonist-antagonist pair of Mckibben muscles}


% author names and affiliations
% use a multiple column layout for up to three different
% affiliations
\author{\IEEEauthorblockN{Ganesh Kumar}
\IEEEauthorblockA{Gepetto\\LAAS-CNRS\\
Toulouse, France\\
Email: gkharish@laas.fr}
\and
\IEEEauthorblockN{Olivier Stasse}
\IEEEauthorblockA{Gepetto\\LAAS-CNRS\\
Toulouse, France\\
Email: stasse@laas.fr}
\and
\IEEEauthorblockN{Bertrand Tondu}
\IEEEauthorblockA{Gepetto\\LAAS-CNRS\\
Toulouse, France\\
Email: tondu@laas.fr\\
}}

% conference papers do not typically use \thanks and this command
% is locked out in conference mode. If really needed, such as for
% the acknowledgment of grants, issue a \IEEEoverridecommandlockouts
% after \documentclass

% for over three affiliations, or if they all won't fit within the width
% of the page, use this alternative format:
% 
%\author{\IEEEauthorblockN{Michael Shell\IEEEauthorrefmark{1},
%Homer Simpson\IEEEauthorrefmark{2},
%James Kirk\IEEEauthorrefmark{3}, 
%Montgomery Scott\IEEEauthorrefmark{3} and
%Eldon Tyrell\IEEEauthorrefmark{4}}
%\IEEEauthorblockA{\IEEEauthorrefmark{1}School of Electrical and Computer Engineering\\
%Georgia Institute of Technology,
%Atlanta, Georgia 30332--0250\\ Email: see http://www.michaelshell.org/contact.html}
%\IEEEauthorblockA{\IEEEauthorrefmark{2}Twentieth Century Fox, Springfield, USA\\
%Email: homer@thesimpsons.com}
%\IEEEauthorblockA{\IEEEauthorrefmark{3}Starfleet Academy, San Francisco, California 96678-2391\\
%Telephone: (800) 555--1212, Fax: (888) 555--1212}
%\IEEEauthorblockA{\IEEEauthorrefmark{4}Tyrell Inc., 123 Replicant Street, Los Angeles, California 90210--4321}}




% use for special paper notices
%\IEEEspecialpapernotice{(Invited Paper)}




% make the title area
\maketitle


\begin{abstract}
%\boldmath
The paper presents an implementation  of ILQR method to control a joint actuated by agonist-antagonistic pair of  Mckibben artificial muscles. The method is applied to the elbow joint of an anthropomorphic 7dofs arm at LAAS-CNRS. It is then compared to the traditional non-linear control method to justify that ILQR is effective in controlling highly non-linear dynamical system even in the absence of good non-linear model which is anyway hard to model in the case of Mckibben muscles.
\end{abstract}
% IEEEtran.cls defaults to using nonbold math in the Abstract.
% This preserves the distinction between vectors and scalars. However,
% if the journal you are submitting to favors bold math in the abstract,
% then you can use LaTeX's standard command \boldmath at the very start
% of the abstract to achieve this. Many IEEE journals frown on math
% in the abstract anyway.

% Note that keywords are not normally used for peerreview papers.
\begin{IEEEkeywords}
Mckibben Muscle, ILQR, non-linear control, Anthropomorphic Pneumatic arm.
\end{IEEEkeywords}






% For peer review papers, you can put extra information on the cover
% page as needed:
% \ifCLASSOPTIONpeerreview
% \begin{center} \bfseries EDICS Category: 3-BBND \end{center}
% \fi
%
% For peerreview papers, this IEEEtran command inserts a page break and
% creates the second title. It will be ignored for other modes.
\IEEEpeerreviewmaketitle



\section{Introduction}
Pneumatic actuators based on Mckibben artificial muscle is known for their non-linearities and hence pose a great control challenge. The sources for non-linearities varies from  hysteresis, saturation and internal friction between fabrics. However, having inherent compliance and very high power to weight ratio have made us to chose it as a research platform for the study compliance control and exploiting this to perform explosive works like hammering or throwing balls. Modelling the non linearities and getting a good model for the joint actuated by artificial muscles is not an easy task. The aim of the paper is to demonstrate that ILQR is effective in controlling joint actuated by Artificial Mckibben muscle even in the absence of good non-linear mode by considering an approximate linear identified model. The proposed method is then compared to the more traditional non-linear control method adopted for controlling the mckibben muscles like sliding mode controller. 
Introduction of the experimental setup and the dynamical model of our joint actuated by Mckibben muscles is presented in section II. Section III will describe the identified linear model which will be used as an approximation of the non-linear model. ILQR algorithm developd by Li and Todorov \cite{todorov} is presented in section IV. Application of ILQR on the Mckibben model is then explained. Experimental results and future work will be discussed in section V and VI.


% needed in second column of first page if using \IEEEpubid
%\IEEEpubidadjcol

% An example of a floating figure using the graphicx package.
% Note that \label must occur AFTER (or within) \caption.
% For figures, \caption should occur after the \includegraphics.
% Note that IEEEtran v1.7 and later has special internal code that
% is designed to preserve the operation of \label within \caption
% even when the captionsoff option is in effect. However, because
% of issues like this, it may be the safest practice to put all your
% \label just after \caption rather than within \caption{}.
%
% Reminder: the "draftcls" or "draftclsnofoot", not "draft", class
% option should be used if it is desired that the figures are to be
% displayed while in draft mode.
%
%\begin{figure}[!t]
%\centering
%\includegraphics[width=2.5in]{myfigure}
% where an .eps filename suffix will be assumed under latex, 
% and a .pdf suffix will be assumed for pdflatex; or what has been declared
% via \DeclareGraphicsExtensions.
%\caption{Simulation Results}
%\label{fig_sim}
%\end{figure}

% Note that IEEE typically puts floats only at the top, even when this
% results in a large percentage of a column being occupied by floats.


% An example of a double column floating figure using two subfigures.
% (The subfig.sty package must be loaded for this to work.)
% The subfigure \label commands are set within each subfloat command, the
% \label for the overall figure must come after \caption.
% \hfil must be used as a separator to get equal spacing.
% The subfigure.sty package works much the same way, except \subfigure is
% used instead of \subfloat.
%
%\begin{figure*}[!t]
%\centerline{\subfloat[Case I]\includegraphics[width=2.5in]{subfigcase1}%
%\label{fig_first_case}}
%\hfil
%\subfloat[Case II]{\includegraphics[width=2.5in]{subfigcase2}%
%\label{fig_second_case}}}
%\caption{Simulation results}
%\label{fig_sim}
%\end{figure*}
%
% Note that often IEEE papers with subfigures do not employ subfigure
% captions (using the optional argument to \subfloat), but instead will
% reference/describe all of them (a), (b), etc., within the main caption.


% An example of a floating table. Note that, for IEEE style tables, the 
% \caption command should come BEFORE the table. Table text will default to
% \footnotesize as IEEE normally uses this smaller font for tables.
% The \label must come after \caption as always.
%
%\begin{table}[!t]
%% increase table row spacing, adjust to taste
%\renewcommand{\arraystretch}{1.3}
% if using array.sty, it might be a good idea to tweak the value of
% \extrarowheight as needed to properly center the text within the cells
%\caption{An Example of a Table}
%\label{table_example}
%\centering
%% Some packages, such as MDW tools, offer better commands for making tables
%% than the plain LaTeX2e tabular which is used here.
%\begin{tabular}{|c||c|}
%\hline
%One & Two\\
%\hline
%Three & Four\\
%\hline
%\end{tabular}
%\end{table}


% Note that IEEE does not put floats in the very first column - or typically
% anywhere on the first page for that matter. Also, in-text middle ("here")
% positioning is not used. Most IEEE journals use top floats exclusively.
% Note that, LaTeX2e, unlike IEEE journals, places footnotes above bottom
% floats. This can be corrected via the \fnbelowfloat command of the
% stfloats package.

\section{Dynamic model of a joint actuated by Mckibben muscles}
Mckibben muscle has inner tube which inflates under pressure $P$ and braided shell surrounding it translates the inflation into axial contraction force $F$.
\begin{equation}
F(\epsilon, P) = (\pi r_{0}^2)P[a(1-k\epsilon)^2-b]
\end{equation}
Where,
\begin{align*}
\epsilon = \frac{l_{o}-l}{l_{o}}\\
a = \frac{3}{\tan^2 {\alpha_{o}}}\\
b = \frac{1}{\sin^2(\alpha_{o})}
\end{align*}
The model in equation (8) assumes that the inner tube is initially cylindrical with length $l_{o}$, radius $r_{o}$ and textile weave of initial braid angle $\alpha_{o}$. $l$ is the instantaneous length of the muscle. So any Mckibben muscle can be characterized by these three parameters. The empirical parameter $k$ is introduced to overcome the error due to the cylindrical inner tube assumption.
Following the human arm model, a pair of artificial muscles can be setup in antagonistic fashion to drive a chained wheel of radius $R$. The antagonistic model used for our pneumatic device is based on the model presented in \cite{tondumuscle} which gives the expression for generated torque, equilibrium position and restoring force as follows:
\begin{equation}
T = R[F_{1}(\epsilon_{1}, P_{1})-F_{2}(\epsilon_{2}, P_{2})]
\end{equation}
Using equation (8), it can be expressed as
\begin{equation}
T = K_{1}(P_{1}-P_{2}) - K_{2}(P_{1}+P_{2})\theta
\end{equation}
Where, $K_{1}$ and $K_{2}$ depends on parameters $l_{o}$, $r_{o}$ and $\alpha_{o}$. $\theta$ is the angular position of the joint or the angular position of the chained wheel.
Setting equation (10) to zero, equilibrium position can be found.
\begin{equation}
\theta_{eq} = \frac{K_{1}(P_{1}-P_{2})}{K_{2}(P_{1}+P_{2})}
\end{equation}
and the restoring torque can be expressed as
\begin{equation}
T_{r} = -K_{2}(P_{1}+P_{2})\theta
\end{equation}
From the above model, it is evident that the system is double input $(P_{1}, P_{2})$ and single output $(\theta)$. However, the objective is to explore the possibility of controlling the stiffness which could be expressed as
\begin{equation}
\sigma = -\frac{T_{r}}{\partial \theta} = K_{2}(P_{1}+P_{2})
\end{equation}
So, the system could  be considered as MIMO (double input $(P_{1}, P_{2})$ and double output $(\theta, \sigma)$.
In the present paper, however, a simplified SISO system model is considered by the following assumptions. 
$P_{m} = P_{1} + P_{2}$ and $\delta P = P_{1} - P_{2}$. The optimal control capabilities would be explored for the applications like achieving a final position with maximum speed or in minimum time or energy.
 

\section{Approximate linear model identification}
The Pneumatic arm is highly non-linear and its very difficult to model these non-linearities. Also some mechanical parameters like friction coefficients and inertias of links are not available. A system identification is carried out to find an approximate linear model using MATLAB system identification toolbox. A set of input is given to the system in open loop and its transient behaviour is recorded to create input-output pair. The input is the pressure variation $\delta P$ and the output is elbow's angular position. Few such sets have been collected. Using system identification toolbox, a linear model has been chosen. It has three poles and no zeros. Its transfer function can be represented as in equation (\cite{eq}). 
The following figure compares the step response of the real system with the identified model.
\begin{figure}[ht!]
\centering
\includegraphics[width=80mm]{identifiedmodel.jpg}
\caption{Step response of elbow joint (green) and identified model (red)\label{Step response}}
\end{figure}
The muscles are first initialised with some pressure before sending a step command that justifies the non-zero initial position in the plot. The transfer function of the identified model is given by the following equation.
\begin{equation}
 H(s) = \frac{207.1}{s^3 + 10.73s^2 + 84.23s +366 }
\end{equation}
\begin{equation}
 \theta(s) = H(s)* (\delta P(s))
\end{equation}

The experiment is repeated at several operating point $(x_{i}, u_{i})$. At each operating point both the muscles is initialized with some pressure, however, sum of the pressure in muscle1 and muscle2 is kept constant at 4.2 bar which is just below the maximum capacity of our intensity pressure converter. We have adopted a transfer function model to find a linear model at every operating point. In order to fit to a transfer function model, which assumes zero initial state conditions, input-output data pair is created by offsetting the initial values at $t=0$. 
 \begin{equation}
 output(t) = x_{i}(t) - x_{i}(0); 
 input(t) = u_{i}(t) - u_{i}(0)
\end{equation}
This $input output$ time domain data pair is used in identification toolbox to find a transfer function of linear model. To simulate the identified model response  and compare with the real system response, lsim function in matlab is used as follows:
\begin{equation}
Ir_{i}(t) = x_{i}(0) + lsim(Imodel_{i},input(t), output(t); 
\end{equation}
Where, $Ir$ is the identified response of the identified model $I_{model}$ at $i$ operating point.


In this way, we have a linear identified model at each operating point. In our experiments we have ten operating points. It is observed that for lower initial pressure, the system behaves linear since one single transfer function is able to fit the system response at several operating points. However, at higher initial pressure the system seems to be strictly non-linear. 

\subsection{Iterative Linear Quadratic Regulator (iLQR)}\label{subsec:ilqr}


Let us consider the discretized version of the system dynamics of Eq. \eqref{}:
\begin{equation}\label{eq:nl_dis_dynamics}
% \vx{x}_{k+1} = f\left(\vx{x}_{k},\vx{u}_{k}\right)
\end{equation}
with, the cost function is defined as,
\begin{equation}\label{eq:cost_fun}
\begin{split}
J_{0}=\frac{1}{2}\left(\vx{x}_N-\vx{x}^{*}\right)^TQ_{f}(\vx{x}_N-\vx{x}^{*})\,+\\
\frac{1}{2}\sum_{k=0}^{N-1}\left(\vx{x}_{k}^TQ\vx{x}_{k}+\vx{u}_{k}^TR\vx{u}_{k}\right)
\end{split}
\end{equation}
where,
\begin{itemize}
\item{$\vx{x}_N$ describes the final state after each execution of the input $\vx{u}_k$.}
\item{$\vx{x}^{*}$ is the given target state.}
\item{$Q$ and $Q_f$ are the state cost weighting matrices.}
\item{$R$ is the control-cost weighting matrix.}
\end{itemize}
Using the the equations \eqref{eq:nl_dis_dynamics} and \eqref{eq:cost_fun}, the iLQR approach proceeds iteratively by obtaining a nominal open loop trajectory $\vx{x}_k$ by applying an input $\vx{u}_k$. With an initial input sequence $\vx{u}_{k}=0$, each iteration produces an improved $\vx{u}_k$ by linearizing the system dynamics around the sequence $\left(\vx{x}_k,\,\vx{u}_k\right)$ and solving a modified LQR problem. The iterations continue until a cost convergence is achieved.

\begin{algorithm}
\caption{ILQR (uk, xi, xf, Qf, Q, R, dt, nIter)}
\begin{algorithmic}
\label{alg:ilqr}
\STATE $cost_{curr}\leftarrow\texttt{ILQRCost}\,\left(\vx{u}_k,\,\vx{x}_i,\, \vx{x}_f,\,Q_f,\,Q,\,R,\,dt \right)$
\FOR{$j=1$,\ldots\,$n_\text{Iter}$}
\STATE $\delta\vx{u}\leftarrow\texttt{ILQRIterate}\,\left(\vx{u}_k,\,\vx{x}_i,\, \vx{x}_f,\,Q_f,\,Q,\,R,\,dt \right)$
\STATE $\vx{u}_k^{'}\leftarrow \vx{u}_k+\alpha\,\delta\vx{u}_k$
\STATE $cost_{new}\leftarrow\texttt{ILQRCost}\,\left(\vx{u}_k^{'},\,\vx{x}_i,\, \vx{x}_f,\,Q_f,\,Q,\,R,\,dt \right)$
\IF{$cost_{new} - cost_{curr} < cost_{threshold}$}
\STATE$\texttt{TerminateIteration}$
\ENDIF
\STATE $cost_{curr}\leftarrow cost_{new}$
\STATE $\vx{u}_{k}\leftarrow \vx{u}_{k}^{'}$
\ENDFOR
\RETURN $\left(\vx{u}_k^{opt},\,cost^{opt}\right)$
\end{algorithmic}
\end{algorithm}
In Algorithm \ref{alg:ilqr}, the \texttt{ILQRCost} procedure, computes the cost of the nominal trajectory using \eqref{eq:cost_fun}. The \texttt{ILQRIterate} implements the linearization procedure mentioned earlier. Eventually, once the cost converges, the \texttt{ILQR} procedure returns the optimal cost-to-go between the states $\vx{x}_{i}$ and $\vx{x}_{f}$.


\section{Experimental results }
The above proposed algorithm is used to control the model identified in section III. The goal is to make the system achieve  final position with the minimum amount of energy spend.  Here let the final position be $Xf = 2.0$ starting from zero initial position. The figures below show the optimal trajectory give by ILQR in time domain and in the phase space. The system  converges within 40 iterations in 200 steps with sampling interval of 0.01s.
\begin{figure}[ht!]
\centering
\includegraphics[width=80mm]{ilqrresponse2.jpg}
\caption{Optimal trajectory give by ILQR for larger steps\label{Optimal trajectory large}}
\end{figure}

\begin{figure}[ht!]
\centering
\includegraphics[width=80mm]{ilqrphaseresponse2.jpg}
\caption{Optimal trajectory in phase space \label{Optimal trajectory phase space}}
\end{figure}
 
The system is able to converge even for 8 steps when sampling interval for ILQR is chosen to be 0.25.
The above algorithm is not input constrained by itself so the above choice of sampling interval and number of steps is made to keep the input change in pressure within the limit (2.5 bar). 
\begin{figure}[ht!]
\centering
\includegraphics[width=80mm]{ilqrresponse1.jpg}
\caption{Optimal trajectory given by ILQR \label{Optimal trajectory}}
\end{figure}

\begin{figure}[ht!]
\centering
\includegraphics[width=80mm]{ilqrphaseresponse1.jpg}
\caption{Optimal trajectory in phase space \label{Optimal trajectory}}
\end{figure}
\section{Comparison with model based controller}

\section{Conclusion and future work}






% if have a single appendix:
%\appendix[Proof of the Zonklar Equations]
% or
%\appendix  % for no appendix heading
% do not use \section anymore after \appendix, only \section*
% is possibly needed

% use appendices with more than one appendix
% then use \section to start each appendix
% you must declare a \section before using any
% \subsection or using \label (\appendices by itself
% starts a section numbered zero.)
%


\appendices
\section{}


% use section* for acknowledgement
\section*{Acknowledgment}


The authors would like to thank...


% Can use something like this to put references on a page
% by themselves when using endfloat and the captionsoff option.
\ifCLASSOPTIONcaptionsoff
  \newpage
\fi



% trigger a \newpage just before the given reference
% number - used to balance the columns on the last page
% adjust value as needed - may need to be readjusted if
% the document is modified later
%\IEEEtriggeratref{8}
% The "triggered" command can be changed if desired:
%\IEEEtriggercmd{\enlargethispage{-5in}}

% references section

% can use a bibliography generated by BibTeX as a .bbl file
% BibTeX documentation can be easily obtained at:
% http://www.ctan.org/tex-archive/biblio/bibtex/contrib/doc/
% The IEEEtran BibTeX style support page is at:
% http://www.michaelshell.org/tex/ieeetran/bibtex/
%\bibliographystyle{IEEEtran}
% argument is your BibTeX string definitions and bibliography database(s)
%\bibliography{IEEEabrv,../bib/paper}
%
% <OR> manually copy in the resultant .bbl file
% set second argument of \begin to the number of references
% (used to reserve space for the reference number labels box)
\begin{thebibliography}{1}
\bibitem{todorov}
L. WeiWei, E.Todorov, ""
\bibitem{tondumuscle}
B.Tondu, S. Ippolito, J. Guiochet, "A Seven-degrees-of-freedom Robot-arm Driven by Pneumatic Artificial Muscles for Humanoid Robots," The International Journal of Robotics Research Vol. 24, No. 4, April 2005, pp. 257-274
\end{thebibliography}

% biography section
% 
% If you have an EPS/PDF photo (graphicx package needed) extra braces are
% needed around the contents of the optional argument to biography to prevent
% the LaTeX parser from getting confused when it sees the complicated
% \includegraphics command within an optional argument. (You could create
% your own custom macro containing the \includegraphics command to make things
% simpler here.)
%\begin{biography}[{\includegraphics[width=1in,height=1.25in,clip,keepaspectratio]{mshell}}]{Michael Shell}
% or if you just want to reserve a space for a photo:

\begin{IEEEbiography}[{\includegraphics[width=1in,height=1.25in,clip,keepaspectratio]{picture}}]{John Doe}
\blindtext
\end{IEEEbiography}

% You can push biographies down or up by placing
% a \vfill before or after them. The appropriate
% use of \vfill depends on what kind of text is
% on the last page and whether or not the columns
% are being equalized.

%\vfill

% Can be used to pull up biographies so that the bottom of the last one
% is flush with the other column.
%\enlargethispage{-5in}




% that's all folks
\end{document}


